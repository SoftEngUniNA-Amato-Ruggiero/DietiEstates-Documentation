\section{Tecnologie utilizzate}
Si è deciso di utilizzare un DBMS relazionale dotato di GIS, per facilitare l'uso di query geospaziali e le ricerche basate sulla distanza. L'uso di un database relazionale garantisce le proprietà ACID (Atomicity, Consistency, Isolation, and Durability).
\newline
Come DBMS è stato scelto \href{https://www.postgresql.org/}{Postgres}, uno dei DBMS relazionali gratuiti e open source piú utilizzati, dotato di un'eccellente estensione spaziale, \href{https://postgis.net/}{PostGIS}. 

\section{Schema}
Le entità individuate sono le seguenti:
\begin{itemize}
    \item TODO
\end{itemize}

Per il sistema di tag è stata definita un'associazione molti a molti tra le inserzioni e i tag presenti nel database. Questo evita duplicazioni e permette di effettuare piú facilmente la ricerca.

\begin{figure}
    \centering
    \includegraphics[width=0.95\linewidth]{Figures/UML/db-public-schema.png}
    \caption{Schema del database}
    \label{fig:db-public-schema}
\end{figure}
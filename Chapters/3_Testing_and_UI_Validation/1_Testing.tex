\section{Metodologie di testing}
Per massimizzare la copertura di righe di codice testate, è stata adottata la metodologia di test white box\footnote{Essendo la complessità del testing white box legata al numero di branch piuttosto che a quello di parametri di input, si è deciso di presentare come casi di test non banali metodi con almeno un branch.}, con approccio branch coverage.
% TODO: verifica che non sia addirittura un modified condition coverage (testa tutte le condizioni che possono portare a un branch, non solo una condizione che esegue il branch)
Tale tecnica puó costringere il programmatore che effettua una modifica al codice a modificare anche i test, riducendo la manutenibilità, ma risulta in accordo con il principio \textit{clean as you code}, che prevede la scrittura di test per ogni aggiunta al codice.
\newline
Sono stati eseguiti test unitari per ogni componente del software in isolamento, utilizzando mockito per simulare le dipendenze, eccezion fatta per il controller.
\newline
Per quest'ultimo sono stati realizzati test di integrazione, sfruttando una migrazione di Flyway per popolare il database con un set di dati iniziali in uno schema apposito e l'iniezione automatica delle dipendenze tramite il framework Spring, salvo per i casi in cui era necessario utilizzare API di un provider esterno (come quello delle notifiche), per i quali si è deciso di utilizzare comunque dei mock.
\newline 
Complessivamente, la coverage dei test copre l'80\% di righe di codice.
Si riportano a scopo esemplificativo alcuni casi di test.

\section{Test cases}

\subsection{promoteToBusinessUser}
Test suite per il metodo di UserPromotionService che si occupa della logica di promozione di utenti da BaseUser a BusinessUser.

\begin{lstlisting}[breaklines=true]
public BusinessUser promoteToBusinessUser(@NonNull BaseUser user, @NonNull RealEstateAgency agency) {
    BusinessUser businessUser;
    Optional<BusinessUser> businessUserOpt = businessUserRepository.findById(user.getId());
    if (businessUserOpt.isEmpty()) {
        businessUser = businessUserRepository.save(new BusinessUser(user, agency));
    } else {
        businessUser = businessUserOpt.get();
        if (!businessUser.getAgency().equals(agency)) {
            throw new ResponseStatusException(HttpStatus.CONFLICT, "User works for a different agency");
        }
    }
    return businessUser;
}
\end{lstlisting}


\begin{figure}[H]
    \centering
    \includegraphics[width=0.8\linewidth]{Figures/CFG/promoteToBusinessUser.png}
    \caption{ CFG - promoteToBusinessUser }
    \label{fig:promoteToBusinessUser}
\end{figure}

Sono stati definiti i seguenti test per coprire i branch del \hyperref[fig:promoteToBusinessUser]{CFG}:
\begin{itemize}
    \item promoteToBusinessUser\_WhenBusinessUserIsEmpty
    \item promoteToBusinessUser\_WhenBUserFoundAgencyMatches
    \item promoteToBusinessUser\_WhenBUserFoundAgencyNotMatch
\end{itemize}

\begin{lstlisting}[breaklines=true]
@Test
void promoteToBusinessUser_WhenBusinessUserIsEmpty() {
    Mockito.when(businessUserRepository.findById(user.getId()))
            .thenReturn(Optional.empty());
    Mockito.when(businessUserRepository.save(Mockito.any(BusinessUser.class)))
            .thenAnswer(invocation -> invocation.getArgument(0));

    BusinessUser businessUser = userPromotionService.promoteToBusinessUser(user, agency);

    assertAll(
            () -> assertEquals(user.getUsername(), businessUser.getUsername()),
            () -> assertEquals(user.getCognitoSub(), businessUser.getCognitoSub()),
            () -> assertEquals(agency, businessUser.getAgency())
    );
}

@Test
void promoteToBusinessUser_WhenBUserFoundAndAgencyMatches() {
    Mockito.when(businessUserRepository.findById(user.getId()))
            .thenReturn(Optional.of(new BusinessUser(user, agency)));

    BusinessUser businessUser = userPromotionService.promoteToBusinessUser(user, agency);

    assertAll(
            () -> assertEquals(user.getUsername(), businessUser.getUsername()),
            () -> assertEquals(user.getCognitoSub(), businessUser.getCognitoSub()),
            () -> assertEquals(agency, businessUser.getAgency())
    );
}

@Test
void promoteToBusinessUser_WhenBUserFoundAgencyNotMatch() {
    RealEstateAgency differentAgency = new RealEstateAgency("11111111", "Different Agency");
    Mockito.when(businessUserRepository.findById(user.getId()))
            .thenReturn(Optional.of(new BusinessUser(user, differentAgency)));

    assertThrows(ResponseStatusException.class,
            () -> userPromotionService.promoteToBusinessUser(user, agency)
    );
}
\end{lstlisting}


\subsection{publishMessageToTopic}
Test suite per il metodo di NotificationsServiceImpl che si occupa di postare un messaggio sul topic delle notifiche.

\begin{lstlisting}[breaklines=true]
public void publishMessageToTopic(@NonNull String message,
                                  Map<String, String> attributes) {
    if (attributes == null) {
        attributes = Collections.emptyMap();
    }

    Map<String, MessageAttributeValue> messageAttributes = attributes.entrySet().stream()
            .collect(Collectors.toMap(Map.Entry::getKey, NotificationsServiceImpl::buildMessageAttribute));

    snsService.withClient(client -> snsService.pubTopic(client, message, messageAttributes));
}
\end{lstlisting}

\begin{figure}[H]
    \centering
    \includegraphics[width=0.8\linewidth]{Figures/CFG/publishMessageToTopic.png}
    \caption{ CFG - publishMessageToTopic }
    \label{fig:publishMessageToTopic}
\end{figure}

Sono stati definiti i seguenti test per coprire i branch del \hyperref[fig:publishMessageToTopic]{CFG}:
\begin{itemize}
    \item publishMessageToTopic
    \item publishMessageToTopic\_WhenAttributesIsNull
\end{itemize}

\begin{lstlisting}[breaklines=true]
@Test
void publishMessageToTopic() {
    String message = "message";
    Map<String, String> attributes = Map.of("key", "value");

    AtomicBoolean pubTopicCalled = new AtomicBoolean(false);
    Mockito.when(snsService.pubTopic(Mockito.eq(snsClient), Mockito.any(), Mockito.any()))
            .thenAnswer(i -> {
                pubTopicCalled.set(true);
                return null;
            });

    notificationsService.publishMessageToTopic(message, attributes);

    assertTrue(pubTopicCalled.get());
}

@Test
void publishMessageToTopic_WhenAttributesIsNull() {
    String message = "message";

    AtomicBoolean pubTopicCalled = new AtomicBoolean(false);
    Mockito.when(snsService.pubTopic(Mockito.eq(snsClient), Mockito.any(), Mockito.any()))
            .thenAnswer(i -> {
                pubTopicCalled.set(true);
                return null;
            });

    notificationsService.publishMessageToTopic(message, null);

    assertTrue(pubTopicCalled.get());
}
\end{lstlisting}


\subsection{enableEmailSubscription}
Test suite per il metodo di NotificationsServiceImpl che si occupa di attivare la sottoscrizione email di un utente ai messaggi postati sul topic delle notifiche.

\begin{lstlisting}[breaklines=true]
public void enableEmailSubscription(@NonNull NotificationsPreferences prefs) throws EmailNotificationsPreferencesUpdateException {
    if (prefs.isEmailNotificationsEnabled()) {
        return;
    }
    String email = prefs.getUser().getUsername();

    snsService.withClient(client -> {
        SubscribeResponse res = snsService.subEmail(client, email);
        if (res.sdkHttpResponse().isSuccessful()) {
            prefs.setSubscriptionArn(res.subscriptionArn());
            log.info("User {} subscribed to email notifications", email);
        } else {
            log.error("Error subscribing to email notifications: {}", res.sdkHttpResponse());
            throw new EmailNotificationsPreferencesUpdateException("Error subscribing to email notifications");
        }
    });
}
\end{lstlisting}

\begin{figure}[H]
    \centering
    \includegraphics[width=0.5\linewidth]{Figures/CFG/enableEmailSubscription.png}
    \caption{ CFG - enableEmailSubscription }
    \label{fig:enableEmailSubscription}
\end{figure}

Sono stati definiti i seguenti test per coprire i branch del \hyperref[fig:enableEmailSubscription]{CFG}:
\begin{itemize}
    \item enableEmailSubscription
    \item enableEmailSubscription\_WhenUnsuccesfulResponse
    \item enableEmailSubscription\_WhenAlreadyEnabled
\end{itemize}

\begin{lstlisting}[breaklines=true]
@Test
void enableEmailSubscription() {
    BaseUser user = new BaseUser("email", "sub");
    NotificationsPreferences prefs = new NotificationsPreferences(user);
    prefs.setSubscriptionArn("");

    SubscribeResponse res = buildSuccesfulSubscribeResponse();

    Mockito.when(snsService.subEmail(snsClient, prefs.getUser().getUsername()))
            .thenReturn(res);

    assertDoesNotThrow(() -> notificationsService.enableEmailSubscription(prefs));

    assertAll(
            () -> assertTrue(prefs.isEmailNotificationsEnabled()),
            () -> assertEquals("subscriptionArn", prefs.getSubscriptionArn())
    );
}

@Test
void enableEmailSubscription_WhenUnsuccesfulResponse() {
    BaseUser user = new BaseUser("email", "sub");
    NotificationsPreferences prefs = new NotificationsPreferences(user);
    prefs.setSubscriptionArn("");

    SubscribeResponse res = buildUnsuccesfulSubscribeResponse();

    Mockito.when(snsService.subEmail(snsClient, prefs.getUser().getUsername()))
            .thenReturn(res);

    assertThrows(EmailNotificationsPreferencesUpdateException.class, () -> notificationsService.enableEmailSubscription(prefs));

    assertAll(
            () -> assertFalse(prefs.isEmailNotificationsEnabled()),
            () -> assertEquals("", prefs.getSubscriptionArn())
    );
}

@Test
void enableEmailSubscription_WhenAlreadyEnabled() {
    BaseUser user = new BaseUser("email", "sub");
    NotificationsPreferences prefs = new NotificationsPreferences(user);
    prefs.setSubscriptionArn("subscriptionArn");

    assertDoesNotThrow(() -> notificationsService.enableEmailSubscription(prefs));

    assertAll(
            () -> assertTrue(prefs.isEmailNotificationsEnabled()),
            () -> assertEquals("subscriptionArn", prefs.getSubscriptionArn())
    );
}
\end{lstlisting}


\subsection{disableEmailSubscription}
Test suite per il metodo di NotificationsServiceImpl che si occupa di disattivare la sottoscrizione email di un utente ai messaggi postati sul topic delle notifiche.

\begin{lstlisting}[breaklines=true]
public void disableEmailSubscription(@NonNull NotificationsPreferences prefs) throws EmailNotificationsPreferencesUpdateException {
    if (!prefs.isEmailNotificationsEnabled()) {
        return;
    }
    String subscriptionArn = prefs.getSubscriptionArn();

    snsService.withClient(client -> {
        UnsubscribeResponse res = snsService.unsubEmail(client, subscriptionArn);
        if (res.sdkHttpResponse().isSuccessful()) {
            prefs.setSubscriptionArn("");
            log.info("User {} unsubscribed from email notifications", prefs.getUser().getUsername());
        } else {
            log.error("Error unsubscribing from email notifications: {}", res.sdkHttpResponse());
            throw new EmailNotificationsPreferencesUpdateException("Error unsubscribing from email notifications");
        }
    });
}
\end{lstlisting}

\begin{figure}[H]
    \centering
    \includegraphics[width=0.5\linewidth]{Figures/CFG/disableEmailSubscription.png}
    \caption{ CFG - disableEmailSubscription }
    \label{fig:disableEmailSubscription}
\end{figure}

Sono stati definiti i seguenti test per coprire i branch del \hyperref[fig:disableEmailSubscription]{CFG}:
\begin{itemize}
    \item disableEmailSubscription
    \item disableEmailSubscription\_WhenUnsuccesfulResponse
    \item disableEmailSubscription\_WhenAlreadyDisabled
\end{itemize}

\begin{lstlisting}
@Test
void disableEmailSubscription() {
    BaseUser user = new BaseUser("email", "sub");
    NotificationsPreferences prefs = new NotificationsPreferences(user);
    prefs.setSubscriptionArn("subscriptionArn");

    UnsubscribeResponse res = buildSuccesfulUnsubscribeResponse();

    Mockito.when(snsService.unsubEmail(snsClient, prefs.getSubscriptionArn()))
            .thenReturn(res);

    assertDoesNotThrow(() -> notificationsService.disableEmailSubscription(prefs));

    assertAll(
            () -> assertFalse(prefs.isEmailNotificationsEnabled()),
            () -> assertEquals("", prefs.getSubscriptionArn())
    );
}

@Test
void disableEmailSubscription_WhenUnsuccesfulResponse() {
    BaseUser user = new BaseUser("email", "sub");
    NotificationsPreferences prefs = new NotificationsPreferences(user);
    prefs.setSubscriptionArn("subscriptionArn");

    UnsubscribeResponse res = buildUnsuccesfulUnsubscribeResponse();

    Mockito.when(snsService.unsubEmail(snsClient, prefs.getSubscriptionArn()))
            .thenReturn(res);

    assertThrows(EmailNotificationsPreferencesUpdateException.class, () -> notificationsService.disableEmailSubscription(prefs));

    assertAll(
            () -> assertTrue(prefs.isEmailNotificationsEnabled()),
            () -> assertEquals("subscriptionArn", prefs.getSubscriptionArn())
    );
}

@Test
void disableEmailSubscription_WhenAlreadyDisabled() {
    BaseUser user = new BaseUser("email", "sub");
    NotificationsPreferences prefs = new NotificationsPreferences(user);
    prefs.setSubscriptionArn("");

    assertDoesNotThrow(() -> notificationsService.disableEmailSubscription(prefs));

    assertAll(
            () -> assertFalse(prefs.isEmailNotificationsEnabled()),
            () -> assertEquals("", prefs.getSubscriptionArn())
    );
}
\end{lstlisting}

\newpage
\section*{Personas}

Per l’analisi dei requisiti è fondamentale identificare con precisione il target di riferimento dell’applicazione. Questo permette di comprendere le esigenze specifiche degli utenti e di progettare un prodotto software in grado di soddisfare le loro aspettative in termini di qualità e usabilità.\newline
L’ampia platea di utenti a cui si rivolge l’applicativo comprende sia persone in cerca di una sistemazione — stabile o temporanea — sia professionisti che lo impiegheranno come strumento di lavoro, come agenti immobiliari o gestori di agenzie.
È quindi necessario tenere conto di un pubblico eterogeneo per età, livello di reddito e istruzione, grado di conoscenza del dominio e competenze digitali, che possono variare anche in modo significativo.\newline
Di seguito, vengono presentati alcuni profili di riferimento che rappresentano le categorie di utenti più comuni.

\section*{Alberico Guadagno}
\begin{figure}[H]
    \centering
    \includegraphics[page=1, width=0.9\textwidth]{Figures/Personas.pdf}
    \caption{Persona - Alberico Guadagno}
    \label{fig:persona1}
\end{figure}

\section*{Arnoldo Pani}
\begin{figure}[H]
    \centering
    \includegraphics[page=2, width=0.9\textwidth]{Figures/Personas.pdf}
    \caption{Persona - Arnoldo Pani}
    \label{fig:persona2-innocent-sin}
\end{figure}

\section*{Pasquale Tippi}
\begin{figure}[H]
    \centering
    \includegraphics[page=3, width=0.9\textwidth]{Figures/Personas.pdf}
    \caption{Persona - Pasquale Tippi}
    \label{fig:persona3-reload}
\end{figure}

\section*{Giulia Stella}
\begin{figure}[H]
    \centering
    \includegraphics[page=4, width=0.9\textwidth]{Figures/Personas.pdf}
    \caption{Persona - Giulia Stella}
    \label{fig:persona4-golden}
\end{figure}

\section*{Michele Caselli}
\begin{figure}[H]
    \centering
    \includegraphics[page=5, width=0.9\textwidth]{Figures/Personas.pdf}
    \caption{Persona - Michele Caselli}
    \label{fig:persona5-royal}
\end{figure}

\section*{Conclusioni}
Dalle personas emergono requisiti importanti. Utenti con problemi di vista avranno bisogno di poter ingrandire il testo senza che questo comprometta l'usabilità, e l'interfaccia dovrà essere responsive per venire incontro a questa eventualità. Utenti con priorità diverse vorranno concentrarsi su aspetti diversi dell'applicazione senza essere intralciati dalle altre possibilità. Per esempio, alcuni agenti immobiliari saranno piú interessati all'efficienza delle operazioni di caricamento, altri preferiranno investirci più tempo per dare maggiore risalto alle loro inserzioni. Alcuni utenti eseguiranno numerose ricerche con filtri diversi, con criteri piú o meno stringenti. Tutti dovranno utilizzare l'applicazione senza precedente training, in maniera intuitiva e veloce: anche utenti meno avvezzi all'uso delle moderne tecnologie dovranno essere in grado di utilizzarla in maniera efficace.
In questa sezione vengono elencati tutti i requisiti funzionali e non funzionali individuati durante la fase di analisi. 
Essi sono stati classificati per tipologia e per area di competenza, al fine di agevolarne la lettura e la tracciabilità nel processo di sviluppo.

\section{Requisiti Funzionali}
I seguenti requisiti definiscono il comportamento del sistema e le funzionalità messe a disposizione degli utenti e dei ruoli interni all’applicazione.

\subsection{Accesso e Sicurezza}
\begin{itemize}
    \item L’utente può creare un account personale specificando \textit{nome}, \textit{cognome}, \textit{indirizzo e-mail} e \textit{password}, oppure tramite autenticazione di terze parti.
    \item L’utente può accedere al sistema inserendo e-mail e password o tramite autenticazione di terze parti.
\end{itemize}

\subsection{Gestione agenzie e Ruoli}
\begin{itemize}
    \item Un utente autenticato può registrare una nuova agenzia immobiliare, inserendone il \textit{nome} e l’\textit{IBAN}. Tale azione lo promuove automaticamente al ruolo di manager per la nuova agenzia.
    \item Un manager può promuovere altri utenti al ruolo di manager della propria agenzia specificando l’indirizzo e-mail, purché non siano già occupati presso un'altra agenzia.
    \item Un manager può promuovere altri utenti al ruolo di agente immobiliare della propria agenzia specificando l’indirizzo e-mail, purché non siano già occupati presso un'altra agenzia.
\end{itemize}

\subsection{Gestione inserzioni}
\begin{itemize}
    \item Un agente immobiliare può caricare nuove \textit{inserzioni} di proprietà immobiliari, specificandone la tipologia e la posizione esatta, e aggiungendo informazioni opzionali di vario tipo, tra cui tag e descrizione in formato rich text con immagini.
\end{itemize}

\subsection{Ricerca e Filtri}
\begin{itemize}
    \item L’utente può effettuare \textit{ricerche avanzate} di immobili filtrando per parametri quali: tipologia dell’inserzione, dimensioni, numero di stanze, fascia di prezzo.
    \item L’utente può filtrare i risultati per posizione, specificando un centro e un raggio.
    \item L’utente può salvare le ricerche effettuate.
\end{itemize}

\subsection{Sistema di Notifiche}
\begin{itemize}
    \item L’utente può selezionare categorie di \textit{notifiche} che desidera ricevere dall'applicativo.
    \item L’utente può selezionare le modalità di notifica che preferisce.
\end{itemize}

\section{Requisiti Non Funzionali}
I requisiti non funzionali specificano vincoli e caratteristiche qualitative del sistema, che ne influenzano l'architettura, le prestazioni e la sicurezza.

\subsection{Architettura e Gestione Dati}
\begin{itemize}
    \item Il sistema utilizza un database relazionale per gestire la persistenza dei dati, la loro correttezza e le loro relazioni in maniera efficiente, sicura e affidabile.
    \item Il database deve fornire supporto a dati geospaziali, per consentire il salvataggio e la ricerca efficiente delle inserzioni in base alla loro posizione.
    \item L'accesso al database è gestito unicamente dal backend.
    \item Il backend espone API RESTful per lo scambio di dati.
\end{itemize}

\subsection{Sicurezza e Autenticazione}
\begin{itemize}
    \item L’autenticazione è gestita tramite un provider esterno, per garantire la sicurezza dei dati personali degli utenti.
    \item Il sistema non deve mai conservare le password degli utenti nel proprio database.
    \item Il backend deve verificare la firma dei token JWT tramite le chiavi pubbliche del provider esterno.
    \item Il frontend richiede il JWT al provider esterno, e deve allegarlo come header a ogni richiesta autenticata che intende effettuare al backend.
\end{itemize}

\subsection{Prestazioni e Affidabilità}
\begin{itemize}
    \item Il backend deve garantire un tempo di risposta non superiore a 0.5s per le ricerche di inserzioni.
    \item Il frontend impone un debounce time di 0.5s tra due ricerche. 
    \item Il servizio deve essere disponibile almeno nel 99\% del tempo operativo.
    \item I componenti dell'applicazione sono hostati tramite servizi di cloud computing per garantirne l'elevata disponibilità.
\end{itemize}

\subsection{Usabilità}
\begin{itemize}
    \item L’interfaccia deve essere intuitiva, coerente e conforme ai principi di design responsivo.
    \item La mappa interattiva deve essere facilmente navigabile e integrata nel flusso di ricerca.
    \item Il sistema deve essere accessibile da browser desktop e mobile.
\end{itemize}

\subsection{Manutenibilità ed Estendibilità}
\begin{itemize}
    \item Il backend deve poter eseguire in parallelo su diverse istanze, per favorire la scalabilità orizzontale.
    \item Il codice sorgente deve essere versionato tramite un VCS e hostato su una repository remota.
    \item Lo schema del database è versionato e aggiornato tramite un database migration tool. 
    \item La qualità del codice del backend è sottoposta a inspection da strumenti di analisi automatica. Il flusso di CI/CD esegue l'ispezione a ogni commit.
    \item L'affidabilità del backend è garantita da una test coverage non inferiore all'80\% di righe di codice.
\end{itemize}